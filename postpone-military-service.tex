\documentclass[a4paper,13pt]{article}

\usepackage{etoolbox} % For if-statement support and togglable variable.
\usepackage[top=0.9in, bottom=0.5in, left=1in, right=1in]{geometry} % Set margin of the page.

\usepackage{thaiSupport} % Load common confiuration for Thai.
\usepackage{dottedField} % Special form field.

\setlength{\parindent}{2.4em} % Set paragraph indentation.
\linespread{1.7} % Simulate ~1.5 line spacing.
\newtoggle{IsAuthorizedPersonMale} % Variable for determining sex of authorized person.

% Variables
% Fill in your inforamtion here, and it should appear in the PDF nicely.
% If the width is not enough, you can still manually edit the document.
\newcommand{\CurrDate}{-}
\newcommand{\CurrMonth}{-}
\newcommand{\CurrYear}{-}

\newcommand{\MyFullName}{-}
\newcommand{\YearOfBirth}{-}
\newcommand{\Reason}{-}

\newcommand{\AddrOfWriting}{-}

\newcommand{\AddrNo}{-}
\newcommand{\AddrStreet}{-}
\newcommand{\AddrMoo}{-}
\newcommand{\AddrSubDistrict}{-}
\newcommand{\AddrDistrict}{-}
\newcommand{\AddrProvince}{-}
\newcommand{\AddrZip}{-}

\newcommand{\AuthorizedPersonFullName}{-}
\toggletrue{IsAuthorizedPersonMale}
% \togglefalse{IsAuthorizedPersonMale} % Un-commnet this for female.

\newcommand{\RelationToAuthorizedPerson}{-}
% End of variables.

\begin{document}

\pagestyle{empty} % Empty the header and footer section.
\restoregeometry

% Title
\begin{center}
\Large
\textbf{\underline{หนังสือมอบอำนาจรับหมายเรียกเข้ารับราชการทหาร}} \\
\end{center}

\normalsize
% Contact Area
\begin{flushright}
  \FillLeft{เขียนที่}{\AddrOfWriting}{3.6in} \\
  \Fill{วันที่}{\CurrDate}{0.8in}
  \Fill{เดือน}{\CurrMonth}{1.3in}
  \Fill{พ.ศ.}{\CurrYear}{1.2in}
\end{flushright}

\noindent
เรื่อง\hspace{13pt}มอบอำนาจให้รับหมายเรียกเข้ารับราชการทหาร แทน \\
เรียน\hspace{10pt}นายอำเภอ \\
\indent\hspace{0.5in}\Fill{ข้าพเจ้า นาย}{\MyFullName}{3.5in} ทหารกองเกิน เกิด พ.ศ. \YearOfBirth \\
อายุ 20 ปี ภูมิลำเนาทหารอยู่ \Fill{บ้านเลขที่}{\AddrNo}{1.5in} \Fill{ถนน}{\AddrStreet}{2in} \Fill{หมู่ที่}{\AddrMoo}{1.3in} \\
\Fill{ตำบล}{\AddrSubDistrict}{1.5in} \Fill{อำเภอ}{\AddrDistrict}{1.5in} \Fill{จังหวัด}{\AddrProvince}{1.5in}\\
ไม่สามารถมารับหมายเรียกเข้ารับราชการทหาร ด้วยตัวเองได้ เนื่องจาก \\
\Fill{}{\Reason}{2.6in}
จึงมอบหมายให้ \Fill{\iftoggle{IsAuthorizedPersonMale}{นาย/\sout{นาง}}{\sout{นาย}/นาง}}{\AuthorizedPersonFullName}{2.7in}
\Fill{ซึ่งเป็น}{\RelationToAuthorizedPerson}{3.2in} มารับหมายเรียกเข้ารับราชการทหารแทน ข้าฯ \\
\indent\hspace{0.5in}ขอได้กรุณาพิจารณามอบหมายเรียกเข้ารับราชการทหาร ให้ตามความประสงค์ด้วย \\

\indent\hspace{2.6in}\Fill{(ลงชื่อ)}{\phantom{a}}{2in} \\
\indent\hspace{2.9in}\makebox[1.6in]{ผู้มอบอำนาจ} \\

\noindent
เรียน\hspace{10pt}นายอำเภอ \\
\indent\hspace{0.5in}\Fill{ด้วย นาย}{\MyFullName}{3in} เกิด พ.ศ. \YearOfBirth\ ต้องมาแสดงตนเพื่อรับหมาย \\
เรียกเข้ารับราชการทหาร ภายใน พ.ศ. \CurrYear\ แต่\Fill{นาย}{\MyFullName}{3in} \\
ไม่สามารถมารับหมายเรียกด้วยตนเองได้ \Fill{เนื่องจาก}{\Reason}{3.4in} \\
จึงมอบหมายให้ ข้าฯ \Fill{ซึ่งเป็น}{\RelationToAuthorizedPerson}{3in} มารับหมายเรียกเข้ารับราชการทหารแทน \\
และข้าฯ ขอยืนยันว่าจะนำหมายเรียกเข้ารับราชการทหารไปมอบให้ นาย \\
\Fill{}{\MyFullName}{3in} ให้จงได้ \\

\indent\hspace{2.6in}\Fill{(ลงชื่อ)}{\phantom{a}}{2in} \\
\indent\hspace{2.9in}\makebox[1.6in]{ผู้รับมอบอำนาจ} \\

\noindent
เรียน\hspace{10pt}นายอำเภอ \\
\indent\hspace{5pt}- เพื่อกรุณาทราบ \\
\indent\hspace{5pt}- ตรวจสอบแล้วถูกต้อง เห็นควรมอบหมายเรียกเข้ารับราชการทหาร ให้รับแทนต่อไป \\

\indent\hspace{2.9in}(\makebox[2in]{\Dotfill}) \\
\indent\hspace{3.7in}สัสดีอำเภอ \\
\indent\hspace{3.2in}\makebox[0.5in]{\Dotfill}/\makebox[0.5in]{\Dotfill}/\makebox[0.5in]{\Dotfill}
\end{document}
